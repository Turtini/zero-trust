% styles/pdf-style.tex
% Portable PDF styling for pandoc + xelatex (GitHub Actions friendly)

\usepackage{fontspec}
\usepackage{xcolor}
\usepackage{geometry}
\usepackage{setspace}
\usepackage{microtype}
\usepackage{hyperref}
\usepackage{fvextra}
\usepackage{fancyhdr}
\usepackage{graphicx}

% Page geometry similar to "clean enterprise PDFs"
\geometry{margin=1in}

% Fonts (portable on ubuntu runners)
\setmainfont{Liberation Sans}
\setsansfont{Liberation Sans}
\setmonofont{Liberation Mono}

% Slightly tighter/cleaner defaults
\setlength{\parindent}{0pt}
\setlength{\parskip}{6pt}
\linespread{1.06}

% Links (subtle)
\hypersetup{
  colorlinks=true,
  linkcolor=black,
  urlcolor=black,
  citecolor=black
}

% Code blocks (pandoc uses Verbatim)
\DefineVerbatimEnvironment{Highlighting}{Verbatim}{
  breaksymbolleft={},
  breaklines=true,
  fontsize=\small
}

% Simple header/footer
\pagestyle{fancy}
\fancyhf{}
\lhead{\footnotesize Zero Trust Architecture: Operational Reference}
\rhead{\footnotesize Turtini LLC}
\cfoot{\footnotesize \thepage}
\renewcommand{\headrulewidth}{0.2pt}
\renewcommand{\footrulewidth}{0pt}

% Pandoc sometimes uses Shaded (depending on highlighting)
% Define it safely without extra packages
\usepackage{framed}
\definecolor{shadecolor}{RGB}{245,245,245}
\renewenvironment{Shaded}{\begin{snugshade}}{\end{snugshade}}

